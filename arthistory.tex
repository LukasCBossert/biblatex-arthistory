% arthistory --%
% Copyright (c) 2016 Lukas C. Bossert 
%  
% This work may be distributed and/or modified under the
% conditions of the LaTeX{} Project Public License, either version 1.3
% of this license or (at your option) any later version.
% The latest version of this license is in
%   http://www.latex-project.org/lppl.txt
% and version 1.3 or later is part of all distributions of LaTeX
% version 2005/12/01 or later.
%!TEX program = xelatex
\documentclass[a4paper,
10pt,
ngerman,
english
]{ltxdoc}
\listfiles
\usepackage[oldstyle]{libertine}
\renewcommand*\ttdefault{lmvtt}
\usepackage[
	backend=biber,
	style=arthistory,
]{biblatex}
\renewcommand\bibfont{\normalfont\footnotesize}
\usepackage{metalogo}
\usepackage{hologo}
\usepackage{babel}
\usepackage{coolthms}


\usepackage{chngcntr}

\usepackage[
  autostyle=true,%
]{csquotes}
\usepackage{multicol}
  \setlength{\columnsep}{1.5cm}
  \setlength{\columnseprule}{0.2pt}

\usepackage{xcolor}
\definecolor{artblue}{RGB}{0,65,137}
\definecolor{artgreen}{RGB}{147,193,26}
\definecolor{artgray}{rgb}{0.5,0.5,0.5}
\definecolor{artpurple}{rgb}{0.58,0,0.82}
\definecolor{artbackground}{rgb}{0.95,0.95,0.92}


\usepackage[ 
	headsepline, 
	footsepline,
%	plainfootsepline, 
%markcase=upper, 
automark, 
draft=false,
]{scrlayer-scrpage} 
\pagestyle{scrheadings}
\clearscrheadfoot
	\ihead{\normalfont\footnotesize \texttt{bib}\LaTeX-style \texttt{arthistory \arthistoryversion} \copyright\ by Lukas C. Bossert | Thorsten Kemper}%
\rofoot{\large\footnotesize  \textbf{\sffamily \thepage}}

\usepackage[%
%  flushmargin, %
%  marginal,
  ragged,%
  hang, %
  bottom%
]{footmisc} %Fussnoten



\usepackage{enumitem}
\setlength{\parindent}{0pt}
\setlength{\parskip}{6pt plus 2pt minus 2pt}
\setenumerate[1]{label=(\alph*),leftmargin=*,nolistsep,parsep=\parskip}
\usepackage{changepage}
\makeindex

\defbibheading{empty}{}
\addbibresource{arthistory-examples.bib}
%\usepackage{caption}

\usepackage[
  skins,
  listings,
  breakable
]{tcolorbox}
\lstMakeShortInline{|}

\newtcblisting[
  auto counter,
  list inside=bibexample,
%  number within=subsection,
  crefname={Example}{Examples}
]{bibexample}[2][]{%
  listing only,
  breakable,
  top=0.5pt,
  bottom=0.5pt,
  colback=artgreen!10,
  colframe=artgreen,
    left=5pt,
    right=5pt,
    sharp corners,
  boxrule=0pt,
  bottomrule=2pt,
  toprule=2pt,
  enhanced jigsaw,
  listing options={%style=tcblatex,
    numbers=left,
    numberstyle=\small\color{artblue},
    moredelim={[is][keywordstyle]{@@}{@@}},
    basicstyle=\footnotesize\ttfamily,
    breaklines=true,
    breakautoindent=false,
    breakindent=0pt,
    escapeinside={{*@}{@*}},
  },%
  lefttitle=0pt,
  coltitle=black,
  colbacktitle=artgreen!20,
  fonttitle=\bfseries\footnotesize,
  title={Example \thetcbcounter:  #2}, 
  #1,%  
  borderline north={1pt}{14.4pt}{artgreen,dashed},
}

\newtcblisting[
auto counter,
%crefname={Example}{Examples}
]{code}{%
    listing only,
    breakable,
    top=0.2pt,
    bottom=0.2pt,
    colback=artgreen!10,
    colframe=artgreen,
    left=5pt,
    right=5pt,
      sharp corners,
    boxrule=0pt,
    bottomrule=0pt,
    toprule=0pt,
    enhanced jigsaw,
    listing options={%style=tcblatex,
%        numbers=left,
        numberstyle=\tiny\color{artblue},
        moredelim={[is][keywordstyle]{@@}{@@}},
        basicstyle=\footnotesize\ttfamily,
        breaklines=true,
        breakautoindent=false,
        breakindent=0pt,
        escapeinside={{*@}{@*}},
    },%
    lefttitle=0pt,
    coltitle=artblue,
    colbacktitle=artgreen!10,
%    fonttitle=\bfseries\footnotesize,
%    title={Example \thetcbcounter:  #2}, 
%   #1,%  
%    borderline north={1pt}{14.4pt}{artgreen,dashed},
}

\newtcolorbox{bibbox}[1]{
      breakable,
      top=5pt,
      bottom=5pt,
      colback=artblue!10,
      colframe=artblue,
      left=5pt,
      right=5pt,
        sharp corners,
      boxrule=0pt,
      bottomrule=2pt,
      toprule=2pt,
      enhanced jigsaw,
        lefttitle=0pt,
        coltitle=black,
        colbacktitle=artblue!20,
        fonttitle=\bfseries\footnotesize,
  title={\Cref{#1}},
        borderline north={1pt}{14.4pt}{artblue,dashed},
}


\newtcolorbox{marker}[1][]{
enhanced,
  before skip=2mm,after skip=3mm,
  boxrule=0.4pt,left=5mm,right=2mm,top=1mm,bottom=1mm,
  colback=artbackground,
  colframe=yellow!20!black,
  sharp corners,rounded corners=southeast,arc is angular,arc=3mm,
  underlay={%
    \path[fill=tcbcol@back!80!black] ([yshift=3mm]interior.south east)--++(-0.4,-0.1)--++(0.1,-0.2);
    \path[draw=tcbcol@frame,shorten <=-0.05mm,shorten >=-0.05mm] ([yshift=3mm]interior.south east)--++(-0.4,-0.1)--++(0.1,-0.2);
    \path[fill=red!50!black,draw=none] (interior.south west) rectangle node[white]{\Huge\bfseries !} ([xshift=4mm]interior.north west);
    },
  drop fuzzy shadow,#1
  }
  
\tcbset{examplebox/.style={%
              boxrule=0pt,
              bottomrule=2pt,
              toprule=2pt,
 colframe=artblue,
  colback=artgreen!10,
   coltitle=artgreen!10,%  coltitle=artblue,
  bicolor,
      sharp corners,
  colbacklower=artblue!10,
  fonttitle=\sffamily\bfseries,
  }}



\newtcblisting{example}{%
    before skip=\baselineskip,
examplebox,
breakable,
%  sidebyside,
%%%%text and listing,
listing and text,
}

\newcommand{\printbib}[2][5em]{%
\begingroup
\begin{bibbox}{#2}
\begin{refsection}
\setlength{\labwidthsameline}{#1} 
\nocite{#2}
\printbibliography[heading=none]
\end{refsection}
\end{bibbox}
\endgroup
}

\newcommand{\printbiball}[2][5em]{%
\begingroup
\setlength{\labwidthsameline}{#1} 
\begin{bibbox}{#2}
\begin{itemize}
\begin{refsection}
\begin{footnotesize}
\nocite{#2}%
\item[English:]{\printbibliography[heading=none]}
\item[German:]\foreignlanguage{ngerman}{\printbibliography[heading=none]}
\item[Italian:]\foreignlanguage{italian}{\printbibliography[heading=none]}
\item[French:]\foreignlanguage{french}{\printbibliography[heading=none]}
\item[Spanish:]\foreignlanguage{spanish}{\printbibliography[heading=none]}
\end{footnotesize}
\end{refsection}
\end{itemize}%
\end{bibbox}
\endgroup
}






\usepackage{hyperxmp}
\usepackage{hyperref}
\hypersetup{					% setup the hyperref-package options
	pdftitle={bib\LaTeX-arthistory},	% 	- title (PDF meta)
	pdfsubject={},% 	- subject (PDF meta)
	pdfauthor={Lukas C. Bossert, Thorsten Kemper},	% 	- author (PDF meta)
	pdfauthortitle={},
	pdfcopyright={This work may be distributed and/or modified under the
conditions of the LaTeX Project Public License, either version 1.3
of this license or (at your option) any later version.},
	pdfhighlight=/N,
	pdfdisplaydoctitle=true,
	pdfdate={\the\year-\the\month-\the\day}
	pdflang={de},
	pdfcaptionwriter={Lukas C. Bossert},
	pdfkeywords={biblatex, art history, humanities},
%	pdfproducer={XeLaTeX},
	pdflicenseurl={http://www.latex-project.org/lppl.txt},
	plainpages=false,			% 	- 
  colorlinks   = true, %Colours links instead of ugly boxes
  urlcolor     =  artblue, %Colour for external hyperlinks
  linkcolor    = artblue, %Colour of internal links
  citecolor   = black, %Colour of citations
  	linktoc=page,
  	pdfborder={0 0 0},			% 	-
	breaklinks=true,			% 	- allow line break inside links
	bookmarksnumbered=true,		%
	bookmarksopenlevel=4,
	bookmarksopen=true,		%
	final=true	% = true, nur bei web-Dokument!! (wichtig!!)
}
\usepackage{bookmark}

\crefformat{lstlisting}{#2example\ #1#3}
\Crefformat{lstlisting}{#2Example #1#3}
\crefmultiformat{lstlisting}{#2examples #1#3}{; #2#1#3}{; #2#1#3}{; #2#1#3}
\Crefmultiformat{lstlisting}{#2Examples #1#3}{; #2#1#3}{; #2#1#3}{; #2#1#3}
\Crefrangeformat{lstlisting}{#3Examples #1#4--#5#2#6}
\crefrangeformat{lstlisting}{#3examples #1#4--#5#2#6}


\begin{document}
\title{\texttt{arthistory} -- \\\texttt{bib\LaTeX} for art historians\footnote{The development of the code is done at \url{https://github.com/LukasCBossert/biblatex-arthistory}. 
Comments and criticisms are welcome.}}

\author{Lukas C. Bossert\protect\\%
{\small \href{mailto:lukas@digitales-altertum.de}{lukas@digitales-altertum.de}}
 \and Thorsten Kemper}
 
\date{Version: \arthistoryversion\ (\arthistorydate)
} 
\maketitle
 
 \begin{abstract}
\noindent This citation style covers the citation and bibliography rules of the \enquote*{Kunsthistorisches Institut der Universität Bonn}.
It introduces bibliography entry types for catalogs and features a tabular bibliography, among other things.
Various options are available to change and adjust the outcome according to one's own preferences. 
The style is compatible with English and German.
 \end{abstract}


\begin{multicols}{2}
\footnotesize\parskip=0mm \tableofcontents
\end{multicols}

\section{Introduction}
|arthistory| is a citation style that complies with the citation and bibliography rules of the \enquote*{Kunsthistorisches Institut der Universität Bonn}\footnote{Website: \url{https://www.khi.uni-bonn.de}.} (henceforth KHI). In particular, it introduces
\begin{itemize}
	\item new entry types for exhibition catalogs and inventory catalogs, and
	%\item special citation short forms for catalogs, encyclopediae as well as primary sources, and
	\item a tabular bibliography that lists each entry's citation short form and its full citation, sorted by the citation short forms.
\end{itemize}
The special citation short forms for catalogs, encyclopediae as well as primary sources described in KHI's guideline are being achieved by using |bib|\LaTeX{}'s standard bibliography entry options.

Various options are available to adjust to common practices not covered by the KHI's advised rules.
%Fortsetzen


\section{Installation}
|arthistory| is part of the distributions MiK\TeX{} %\footnote{Website: \url{http://www.miktex.org}.} 
and \TeX{} Live%\footnote{Website: \url{http://www.tug.org/texlive}.}~-- thus,
and you can easily install it using the respective package manager. 
If you would like to
install |arthistory| manually, do the following:
Download the folder |arthistory| with all relevant files from the CTAN-server\footnote{\url{http://mirrors.ctan.org/macros/latex/contrib/biblatex-contrib/arthistory.zip}} and copy it to the \texttt{\$LOCALTEXMF} directory of
 your system.\footnote{If you don't know what that is, have a look at
\url{http://www.tex.ac.uk/cgi-bin/texfaq2html?label=tds} or 
\url{http://mirror.ctan.org/tds/tds.html}.} 
Refresh your filename database.\footnote{ 
Here is some additional information from the UK \TeX\ FAQ:
\begin{itemize}[nosep,after=\vspace{-\baselineskip} ]
  \item \href{%
    http://www.tex.ac.uk/cgi-bin/texfaq2html?label=install-where}{%
    Where to install packages}
  \item \href{%
    http://www.tex.ac.uk/cgi-bin/texfaq2html?label=inst-wlcf}{%
    Installing files \enquote{where \LaTeX{} /TeX\ can find them}}
  \item \href{%
    http://www.tex.ac.uk/cgi-bin/texfaq2html?label=privinst}{%
    \enquote{Private} installations of files}
\end{itemize}
}
%%introduction from biblatex-dw copied and applied. might to be rewritten.


\section{Loading the package}
 \DescribeMacro{arthistory} The name of the bib\LaTeX{} style is |arthistory|. It has to be activated in the preamble.

\begin{code}
\usepackage[style=arthistory,%
          *@\meta{further options}@*]{biblatex}
\bibliography*@\marg{|bib|-file}@*
\end{code}

Without enabling any further options, the style follows the rules of the \enquote*{Kunsthistorisches Institut der Universität Bonn}.
No additional settings are needed,
but you can change the outcome by using some options which are explained below.%\footnote{For an easy and unproblematic compiling we suggest to use \hologo{XeLaTeX} or \hologo{LuaTeX}.}

At the end of your document you can write the command |\printbibliography| to print 
a single bibliography.
However, since |arthistory| supports citation styles for catalogs and primary sources that differ from the standard citations of common scientific contributions, we suggest having them listed in separate bibliographies, see \cref{sec:sepbib}.

%\section{Overview}\label{overview}
%There follows a quick overview of possible options of the style |arthistory|. 
%Furthermore you can -- at your own risk -- also use the conventional |bib|\LaTeX-options relating to indent, etc. 
%For that please see the excellent documentation of  |bib|\LaTeX.


\section{Bibliography entries}
Besides loading |arthistory|, in order to comply with the KHI's bibliography rules, users' actions will be required mostly when entering bibliography items.

\subsection{Entry fields}

\subsubsection{\texttt{arthistory}-specific options}

\DescribeMacro{arthist}
\enquote{H-ArtHist} is a popular newsletter for... %Fortsetzen
|arthist=|\marg{value} will print...

\DescribeMacro{eventsubtitle}
Use this field to specify the subtitle of an exhibition a given |exhibcatalog| is based on.

\DescribeMacro{exhibfirstdate}
Specifies the time span of an exhibition an |Exhibcatalog| is based on. Dates of the first and last day of the respective exhibition are to be entered in the format \meta{year}|-|\meta{month}|-|\meta{day}. A typical entry may be
\begin{code}
@Exhibcatalog{AusstellungBonn2005,
  *@\ldots@*
  exhibfirstdate = {2005-11-25/2006-03-19},
  *@\ldots@*
}.
\end{code}
If an exhibition has more than one date, use |exhibseconddate| and |exhibthirddate| accordingly.

\DescribeMacro{exhibfirstlocation}
For the city in which an exhibition has been shown. E.g.,
\begin{code}
@Exhibcatalog{AusstellungBonn2005,
  *@\ldots@*
  exhibfirstlocation = {Bonn},
  *@\ldots@*
}.
\end{code}
If an exhibition has more than one date, use |exhibsecondlocation| and |exhibthirdlocation| accordingly.

\DescribeMacro{exhibfirstmuseum}
For the venue---usually a museum---where an exhibition is shown. E.g.,
\begin{code}
@Exhibcatalog{AusstellungBonn2005,
  *@\ldots@*
  exhibfirstmuseum = {Kunst- und Ausstellungshalle der Bundesrepublik Deutschland},
  *@\ldots@*
}.
\end{code}
If an exhibition has more than one date, use |exhibsecondmuseum| and |exhibthirdmuseum| accordingly.

\DescribeMacro{thesisdate}
The year when a PhD (or \enquote{Habilitation}) thesis was defended.

\DescribeMacro{thesistype = tzugl}
To be used when a publication (typically a book) is partly based on a submitted PhD (or \enquote{Habilitation}) thesis.


\subsubsection{Important standard \texttt{bib}\LaTeX{} options}\label{sec:bibl-efields}
Here, we list otherwise standard |bib|\LaTeX{} options that are essential to comply by the KHI's bibliography rules.

\DescribeMacro{eventdate}
Use this field to specify the year(s) of an exhibition a given |exhibcatalog| is based on. In case the exhibition covered two subsequent years, enter them as \meta{first year}|/|\meta{second year}. An example would be
\begin{code}
@Exhibcatalog{AusstellungBonn2005,
  *@\ldots@*
  eventdate = {2005/2006},
  *@\ldots@*
}.
\end{code}
Note specifying this field has the sole purpose of generating the correct citation short form.
You will also have to enter the field |exhibfirstdate| (and possibly |exhibseconddate| or even |exhibthirddate|) for the long bibliography entry, and the fields |date| or |year| for the publication itself. 

\DescribeMacro{eventtitle}
Use this field to specify the name of an exhibition a given |exhibcatalog| is based on.

\DescribeMacro{institution}
For the institution where a thesis was defended.

\DescribeMacro{keywords = source}\label{sec:source}
%\DescribeMacro{source}
This option is reserved for entries that are primary sources (e.\,g. Alberti, Paleotti, etc). %cf. \cref{source}.
If enabled, the entry can be listed in a separate bibliography for primary sources. (Actually you don't need to use the term \enquote*{source} -- you can pick any term you like.) Also see \cref{sec:sepbib}.

In addition to that, you should define a |shorthand| that differs from the usual author year citation of regular scientific works in order to comply with the KHI's citation rules.

\DescribeMacro{shorthand}\label{sec:shorthand}
%\DescribeMacro{shorthand}
Entering a shorthand will replace the otherwise automatically generated, document type-appropriate citation short form by the typed-in content.

You need to make use of this option when entering a primary source in your bibliography file (along with the option |keywords = source|). The entry should consist of a short version of the primary source's author's name and (possibly an abbreviation of) their contribution's title.

Here is an example of Casanova's Theory of painting:
\begin{bibexample}[label=CasanovaMalerei]{{@}Book\{CasanovaMalerei,…\}}
@Book{CasanovaMalerei,
  author    = {Casanova, Giovanni Battista},
  editor    = {Kanz, Roland},
  title     = {Theorie der Malerei},
  location  = {München},
  year      = {2008},
  series    = {Phantasos},
  number    = {8},
  shorthand = {Casanova, Theorie der Malerei},
  keywords  = {source},
}
\end{bibexample}
As Casanova's text is being published in a book, in the usual case its citation would automatically consist of the author's name and the year of the publication.
However, you will notice that the citation footnote is an exact copy of the |source| option's content.
\printbib[6em]{CasanovaMalerei}

\DescribeMacro{shorttitle}
Defining a |shorttitle| can be especially useful when encyclopediae whose title starts with an article are being cited.
%Include Grove encyclopedia!

\DescribeMacro{sortkey}
When entering (exhibiton) catalogs.

\DescribeMacro{type}
To specify the type of thesis. Possible values are HABIL,PHDTHESIS


\subsection{Entry types}


\subsubsection{\texttt{arthistory}-specific entry types}\label{sec:arth-etypes}

\DescribeMacro{@catalog}
This entry type marks catalogs of the permanent inventory of a museum's art collection (\enquote{Bestandskatalog}).
\begin{bibexample}[label=KatSORRusche2010]{{@}catalog\{KatSORRusche2010,…\}}
@Catalog{KatSORRusche2010,
  editor   = {Raupp, Hans-Joachim},
  title    = {Historien und Allegorien},
  year     = {2010},
  location = {Münster and Hamburg and London},
  label    = {S{{\O}}R Rusche},
  number   = {4},
  series   = {Niederländische Malerei des 17. Jahrhunderts der S{{\O}}R Rusche-Sammlung},
}
\end{bibexample}
\printbib[6em]{KatSORRusche2010}

\DescribeMacro{@exhibcatalog}
This is for catalogs of temporary exhibitions.
\begin{bibexample}[label=AusstellungBonn2005]{{@}exhibcatalog\{AusstellungBonn2005,…\}}
@Exhibcatalog{AusstellungBonn2005,
  editor          = {Jutta Frings},
  year            = {2005},
  location        = {Leipzig},
  eventdate       = {2005/2006},
  eventsubtitle   = {Kunst und Kultur der Päpste II 1572--1676},
  eventtitle      = {Barock im Vatikan},
  eventtitleaddon = {Bonn, Kunst- und Ausstellungshalle der Bundesrepublik Deutschland, 25. November 2005 bis 19. März 2006; Berlin, Martin-Gropius-Bau, 12. April bis 10. Juli 2006},
  keywords        = {Ausstellung},
  venue           = {Bonn/Berlin},
}
\end{bibexample}
\printbib[6em]{AusstellungBonn2005}


\subsubsection{Standard \texttt{bib}\LaTeX{} types}

\DescribeMacro{@article}
\begin{bibexample}[label=Schlegel1992]{{@}Article\{Schlegel1992,…\}}
@Article{Schlegel1992,
  author       = {Schlegel, Ursula},
  title        = {Ein Terracottamodell des Bartolomeo Ammannati},
  journaltitle = {Paragona/Arte},
  volume       = {43},
  pages        = {25--30},
  year         = {1992},
  number       = {503},
}
\end{bibexample}
\printbib[6em]{Schlegel1992}

\DescribeMacro{@book}
\begin{bibexample}[label=vonrosen2009]{{@}Book\{vonrosen2009,…\}}
@Book{vonrosen2009,
  author   = {von Rosen, Valeska},
  title    = {Caravaggio und die Grenzen des Darstellbaren},
  subtitle = {Ambiguität, Ironie und Performativität in der Malerei um 1600},
  location = {Berlin},
  year     = {2009},
}
\end{bibexample}
\printbib[6em]{vonrosen2009}

\DescribeMacro{@Reference}
This entry type is suited for encyclopediae. 

\DescribeMacro{@Review}
This is for reviews of dissertation or habilitation theses, conference proceedings, other scientific publications, exhibitions etc.
For a full citation of a review you are asked to name the reviewed work in detail.
The following example will show an easy way to combine the review with the reviewed work.
First we have the reviewed work:
\begin{bibexample}[label=Heesen2012]{{@}Book\{Heesen2012,…\}}
@Book{Heesen2012,
  author     = {te Heesen, Anke},
  title      = {Theorien des Museums},
  publisher  = {Junius Verlag},
  location   = {Hamburg},
  year       = {2012},
  titleaddon = {Zur Einführung},
}
\end{bibexample}
followed by the review itself:
\begin{bibexample}[label=Bonnet2013]{{@}Review\{Bonnet2013,…\}}
@Review{@Review{Bonnet2013,
  author       = {Bonnet, Anne-Marie},
  number       = {10},
  volume       = {14},
  journaltitle = {Kunstform},
  related      = {Heesen2012},
  relatedtype  = {reviewof},
  year         = {2013},
  url          = {http://www.arthistoricum.net/kunstform/rezension/ausgabe/2013/10/22240/},
}
\end{bibexample}
You may have noticed that the review (|Bonnet2013|) is connected to the entry |Heesen2012| by the field |related|.
In addition we need to qualify the relation between the connected entries:
This is done with |relatedtype = {reviewof}|.
This so-called |bibstring| is reserved for reviews and contains the translation of \emph{Review of}, e.\,g. \emph{Rezension von} in German, which will be printed in squared brackets.
%You don’t have to type in all relevant information of the reviewed work in the entry of the review, 
%since they will be inserted automatically and dynamically with the  |related|-function. 
%So whenever settings in the reviewed work are changed the print of the review will be automatically adjusted. 
%Furthermore, even if the review is cited, the reviewed work won't be listed in the bibliography until it is explicitly cited in the text.
\printbib[6em]{Bonnet2013}


\section{Preamble options}\label{preamble_options}
In this section we describe options that can be loaded along with |bib|\LaTeX{} in the document preamble. With one exception, every option will lead to a deviation from the rules advised by the KHI guide; several options listed will allow the user to adhere to bibliography practices common in the field. If you do not intend to deviate at all, you can skip this section.

\DescribeMacro{allnamesfamilygiven}
When enabled, last names will precede first names in all instances.

\DescribeMacro{citeauthorformat}
You can chose how the name of authors or editors are displayed within your text when they are cited with \cs{citeauthor}\marg{bibtex-key}.
You can chose between the options \meta{initials}, \meta{full}, \meta{family}, \meta{firstfull}; 
cf. %\cref{citeauthorformat}.

\DescribeMacro{enddot}
When including |enddot=true|, every bibliography entry will end with a dot.

\DescribeMacro{firstcitefull}
With |firstcitefull = true|, the first time (and \emph{only} the first time) a work is being cited in the document, a full citation will be printed.

\DescribeMacro{namelinked}
When included and |hyperref| loaded, both name and year in a short citation will link to the respective bibliography entry.

\DescribeMacro{pagesfull}
When including |pagesfull| in the options, bibliography entries' page numbers will be preceded by \enquote{pp.} (or \enquote{S.} in German). The same holds for citation postnotes if they contain page numbers.

\DescribeMacro{publisher}
The publisher is being listed in the bibliography entries.

\DescribeMacro{width}
|width=|\meta{value} defines the bibliography width between label and reference.


\section{Cite commands}\label{cite-commands}
|arthistory| supports most/all standard |bib|\LaTeX{} citation commands. We refer the reader said package's documentation to learn more about the full set of commands. In the following, we will describe, for users with little experience in \LaTeX{} or |bib|\LaTeX, how standard citation commands can be employed to abide by the KHI's citation rules.

\DescribeMacro{\cite}%
The standard \cs{cite} command invokes a authoryear-style citation without any parentheses. Because of the KHI's requirements, \cs{cite} will typically be invoked from within a footnote:
\begin{code}
\footnote{
  *@\ldots@*
  \cite*@\oarg{prenote}\oarg{postnote}\marg{bibtex-key}@*
  *@\ldots@*
}
\end{code}

\meta{prenote} sets a short preliminary note (e.\,g. \enquote{Vgl.}) and \meta{postnote} is usually used for page numbers.
If only one optional argument is used then it is \oarg{postnote}.
\begin{code}
\footnote{*@\ldots@*\cite*@\oarg{postnote}\marg{bibtex-key}\ldots@*}
\end{code}
The \meta{bibtex-key} corresponds to the key from the bibliography file.

\DescribeMacro{\footcite}
The same as manually adding a footnote first and \cs{cite} subsequently can be achieved in one step via the \cs{footcite} command:
\begin{code}
\footcite*@\oarg{prenote}\oarg{postnote}\marg{bibtex-key}%@*
\end{code}
This command will be useful if nothing more than a citation with very short prenotes and/or postnotes is needed. When a citation is embedded in a text paragraph, the former combination of \cs{footnote} and \cs{cite} is advisable.

As noted above, all the well-documented citation commands of the |bib|\LaTeX{} package are supported.
\DescribeMacro{\cites}
E.g., if one wants to cite several authors or works a very convenient way is the following, using the \cs{cites}-command (typically in a footnote):
\begin{code}
\cites(pre-prenote)(post-postnote)
  *@\oarg{prenote}\oarg{postnote}\marg{bibtex-key}@*%
  *@\oarg{prenote}\oarg{postnote}\marg{bibtex-key}@*%
  *@\oarg{prenote}\oarg{postnote}\marg{bibtex-key}\ldots@*
\end{code}
Other examples are \cs{parencite},\cs{textcite} and their multi-entry alternatives, and commands such as \cs{citeauthor} and \cs{citetitle}.

\DescribeMacro{smartcite}\DescribeMacro{autocite}
Note that \cs{smartcite} and \cs{autocite} behave a little bit differently than in \enquote{standard} |bib|\LaTeX{} styles. When appearing in a footnote, both commands will behave as |arthistory|'s \cs{cite} rather than \cs{parencite}. In addition to that, by default \cs{autocite} appearing in the text body behaves like \cs{footcite}.


\section{Separate bibliographies}\label{sec:sepbib}
Here, we describe how you can list separate bibliographies for primary sources and secondary literature (and possibly catalogs as well). This can be achieved by standard |bib|\LaTeX{} procedures; experienced users will want to skip this section.

You may have noticed that we listed the option |keywords = {source}| in \cref{bibl-efields} and that we used it in \cref{sec:arth-etypes}.

%\DescribeMacro{\parencite}
%Sometimes a citation has to be put in parentheses. 
%Therefore we implemented the command \cs{parencite}:
%\begin{code}
%\parencite*@\oarg{postnote}\marg{bibtex-key}%@*
%\end{code} 
%This cite command takes care of the correct corresponding parentheses and brackets.
%Especially in |@Inreference| citations the parentheses  change to (square) brackets.
%
%
%\DescribeMacro{\parencites}
%Of course there is also the possibility to cite several authors/works in parentheses.
%This is done with \cs{parencites}:
%\begin{code}
%\parencites(pre-prenote)(post-postnote)%
%*@\oarg{prenote}\oarg{postnote}\marg{bibtex-key}@*%
%*@\oarg{prenote}\oarg{postnote}\marg{bibtex-key}@*%
%*@\oarg{prenote}\oarg{postnote}\marg{bibtex-key}\ldots@*
%\end{code}
 %
%\DescribeMacro{\textcite}
%Beside the listed \cs{cite} commands above there is a third way of citing:
%\cs{textcite} is useful if the author should be mentioned in the text and
%the remaining components such as year and page will immediately follow in parentheses. 
%\begin{code}
%\textcite*@\oarg{postnote}\marg{bibtex-key}@*
%\end{code} 
%
%\DescribeMacro{\textcites}
%And again there is also a \cs{textcites} in case of several authors: 
%\begin{code}
%\textcites(pre-prenote)(post-postnote)%
  %*@\oarg{prenote}\oarg{postnote}\marg{bibtex-key}@*%
  %*@\oarg{prenote}\oarg{postnote}\marg{bibtex-key}@*%
  %*@\oarg{prenote}\oarg{postnote}\marg{bibtex-key}\ldots@*
%\end{code}
%
%
%
%Futhermore you can use following \cs{cite} commands:
%\DescribeMacro{\footcite} 
%\begin{code}
%\footcite*@\oarg{prenote}\oarg{postnote}\marg{bibtex-key}@*
%\end{code} 
%
%\begin{example}
%\footcite{Schlegel1992}
%\end{example}
%
%
%\DescribeMacro{\footcitetext} 
%\begin{code}
%\footcitetext*@\oarg{prenote}\oarg{postnote}\marg{bibtex-key}@*
%\end{code} 
%
%\begin{example}
%\footcitetext{Schlegel1992}
%\end{example}
%
%
%\DescribeMacro{\smartcite} 
%\begin{code}
%\smartcite*@\oarg{prenote}\oarg{postnote}\marg{bibtex-key}@*
%\end{code} 
%
%\begin{example}
%\smartcite{Schlegel1992}
%\end{example}
%
%
%
%
%
%
%\DescribeMacro{\citeauthor}\DescribeMacro{\citetitle}\label{citeauthor}%
%Furthermore and in addition to the ›normal‹ \cs{cite}-commands one can also cite only the author or the work title in the text and in the footnotes.
%\begin{code}
%\citeauthor*@\oarg{prenote}\oarg{postnote}\marg{bibtex-key}@*
%\end{code} 
  %and for the works 
%\begin{code}
%\citetitle*@\oarg{prenote}\oarg{postnote}\marg{bibtex-key}@*
%\end{code} 


\end{document}