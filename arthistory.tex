% arthistory --%
% Copyright (c) 2016 Lukas C. Bossert 
%  
% This work may be distributed and/or modified under the
% conditions of the LaTeX Project Public License, either version 1.3
% of this license or (at your option) any later version.
% The latest version of this license is in
%   http://www.latex-project.org/lppl.txt
% and version 1.3 or later is part of all distributions of LaTeX
% version 2005/12/01 or later.
%!TEX program = xelatex
\documentclass[a4paper,
10pt,
ngerman,
english
]{ltxdoc}
\def\thisarthistoryversion{0.1}
\def\thisarthistorydate{2016-09-25}
\listfiles
\usepackage[oldstyle]{libertine}
\renewcommand*\ttdefault{lmvtt}
\usepackage[
	backend=biber,
	style=arthistory,
]{biblatex}
\renewcommand\bibfont{\normalfont\footnotesize}
\usepackage{metalogo}
\usepackage{hologo}
\usepackage{babel}
\usepackage{coolthms}


\usepackage{chngcntr}

\usepackage[
  autostyle=true,%
]{csquotes}
\usepackage{multicol}
  \setlength{\columnsep}{1.5cm}
  \setlength{\columnseprule}{0.2pt}

\usepackage{xcolor}
\definecolor{artblue}{RGB}{0,65,137}
\definecolor{artgreen}{RGB}{147,193,26}
\definecolor{artgray}{rgb}{0.5,0.5,0.5}
\definecolor{artpurple}{rgb}{0.58,0,0.82}
\definecolor{artbackground}{rgb}{0.95,0.95,0.92}


\usepackage[ 
	headsepline, 
	footsepline,
%	plainfootsepline, 
%markcase=upper, 
automark, 
draft=false,
]{scrlayer-scrpage} 
\pagestyle{scrheadings}
\clearscrheadfoot
	\ihead{\normalfont\footnotesize \texttt{bib}\LaTeX-style \texttt{arthistory \arthistoryversion} \copyright\ by Lukas C. Bossert | Thorsten Kemper}%
\rofoot{\large\footnotesize  \textbf{\sffamily \thepage}}

\usepackage[%
%  flushmargin, %
%  marginal,
  ragged,%
  hang, %
  bottom%
]{footmisc} %Fussnoten



\usepackage{enumitem}
\setlength{\parindent}{0pt}
\setlength{\parskip}{6pt plus 2pt minus 2pt}
\setenumerate[1]{label=(\alph*),leftmargin=*,nolistsep,parsep=\parskip}
\usepackage{changepage}
\makeindex

\defbibheading{empty}{}
\addbibresource{arthistory-examples.bib}
%\usepackage{caption}

\usepackage[
  skins,
  listings,
  breakable
]{tcolorbox}
\lstMakeShortInline{|}

\newtcblisting[
  auto counter,
  list inside=bibexample,
%  number within=subsection,
  crefname={Example}{Examples}
]{bibexample}[2][]{%
  listing only,
  breakable,
  top=0.5pt,
  bottom=0.5pt,
  colback=artgreen!10,
  colframe=artgreen,
    left=5pt,
    right=5pt,
    sharp corners,
  boxrule=0pt,
  bottomrule=2pt,
  toprule=2pt,
  enhanced jigsaw,
  listing options={%style=tcblatex,
    numbers=left,
    numberstyle=\small\color{artblue},
    moredelim={[is][keywordstyle]{@@}{@@}},
    basicstyle=\footnotesize\ttfamily,
    breaklines=true,
    breakautoindent=false,
    breakindent=0pt,
    escapeinside={{*@}{@*}},
  },%
  lefttitle=0pt,
  coltitle=black,
  colbacktitle=artgreen!20,
  fonttitle=\bfseries\footnotesize,
  title={Example \thetcbcounter:  #2}, 
  #1,%  
  borderline north={1pt}{14.4pt}{artgreen,dashed},
}

\newtcblisting[
auto counter,
%crefname={Example}{Examples}
]{code}{%
    listing only,
    breakable,
    top=0.2pt,
    bottom=0.2pt,
    colback=artgreen!10,
    colframe=artgreen,
    left=5pt,
    right=5pt,
      sharp corners,
    boxrule=0pt,
    bottomrule=0pt,
    toprule=0pt,
    enhanced jigsaw,
    listing options={%style=tcblatex,
%        numbers=left,
        numberstyle=\tiny\color{artblue},
        moredelim={[is][keywordstyle]{@@}{@@}},
        basicstyle=\footnotesize\ttfamily,
        breaklines=true,
        breakautoindent=false,
        breakindent=0pt,
        escapeinside={{*@}{@*}},
    },%
    lefttitle=0pt,
    coltitle=artblue,
    colbacktitle=artgreen!10,
%    fonttitle=\bfseries\footnotesize,
%    title={Example \thetcbcounter:  #2}, 
%   #1,%  
%    borderline north={1pt}{14.4pt}{artgreen,dashed},
}

\newtcolorbox{bibbox}[1]{
      breakable,
      top=5pt,
      bottom=5pt,
      colback=artblue!10,
      colframe=artblue,
      left=5pt,
      right=5pt,
        sharp corners,
      boxrule=0pt,
      bottomrule=2pt,
      toprule=2pt,
      enhanced jigsaw,
        lefttitle=0pt,
        coltitle=black,
        colbacktitle=artblue!20,
        fonttitle=\bfseries\footnotesize,
  title={\Cref{#1}},
        borderline north={1pt}{14.4pt}{artblue,dashed},
}


\newtcolorbox{marker}[1][]{
enhanced,
  before skip=2mm,after skip=3mm,
  boxrule=0.4pt,left=5mm,right=2mm,top=1mm,bottom=1mm,
  colback=artbackground,
  colframe=yellow!20!black,
  sharp corners,rounded corners=southeast,arc is angular,arc=3mm,
  underlay={%
    \path[fill=tcbcol@back!80!black] ([yshift=3mm]interior.south east)--++(-0.4,-0.1)--++(0.1,-0.2);
    \path[draw=tcbcol@frame,shorten <=-0.05mm,shorten >=-0.05mm] ([yshift=3mm]interior.south east)--++(-0.4,-0.1)--++(0.1,-0.2);
    \path[fill=red!50!black,draw=none] (interior.south west) rectangle node[white]{\Huge\bfseries !} ([xshift=4mm]interior.north west);
    },
  drop fuzzy shadow,#1
  }
  
\tcbset{examplebox/.style={%
              boxrule=0pt,
              bottomrule=2pt,
              toprule=2pt,
 colframe=artblue,
  colback=artgreen!10,
   coltitle=artgreen!10,%  coltitle=artblue,
  bicolor,
      sharp corners,
  colbacklower=artblue!10,
  fonttitle=\sffamily\bfseries,
  }}



\newtcblisting{example}{%
    before skip=\baselineskip,
examplebox,
breakable,
%  sidebyside,
%%%%text and listing,
listing and text,
}

\newcommand{\printbib}[2][5em]{%
\begingroup
\begin{bibbox}{#2}
\begin{refsection}
\setlength{\labwidthsameline}{#1} 
\nocite{#2}
\printbibliography[heading=none]
\end{refsection}
\end{bibbox}
\endgroup
}

\newcommand{\printbiball}[2][5em]{%
\begingroup
\setlength{\labwidthsameline}{#1} 
\begin{bibbox}{#2}
\begin{itemize}
\begin{refsection}
\begin{footnotesize}
\nocite{#2}%
\item[English:]{\printbibliography[heading=none]}
\item[German:]\foreignlanguage{ngerman}{\printbibliography[heading=none]}
\item[Italian:]\foreignlanguage{italian}{\printbibliography[heading=none]}
\item[French:]\foreignlanguage{french}{\printbibliography[heading=none]}
\item[Spanish:]\foreignlanguage{spanish}{\printbibliography[heading=none]}
\end{footnotesize}
\end{refsection}
\end{itemize}%
\end{bibbox}
\endgroup
}






\usepackage{hyperxmp}
\usepackage{hyperref}
\hypersetup{					% setup the hyperref-package options
	pdftitle={bib\LaTeX-arthistory},	% 	- title (PDF meta)
	pdfsubject={},% 	- subject (PDF meta)
	pdfauthor={Lukas C. Bossert, Thorsten Kemper},	% 	- author (PDF meta)
	pdfauthortitle={},
	pdfcopyright={This work may be distributed and/or modified under the
conditions of the LaTeX Project Public License, either version 1.3
of this license or (at your option) any later version.},
	pdfhighlight=/N,
	pdfdisplaydoctitle=true,
	pdfdate={\the\year-\the\month-\the\day}
	pdflang={de},
	pdfcaptionwriter={Lukas C. Bossert},
	pdfkeywords={biblatex, art history, humanities},
%	pdfproducer={XeLaTeX},
	pdflicenseurl={http://www.latex-project.org/lppl.txt},
	plainpages=false,			% 	- 
  colorlinks   = true, %Colours links instead of ugly boxes
  urlcolor     =  artblue, %Colour for external hyperlinks
  linkcolor    = artblue, %Colour of internal links
  citecolor   = black, %Colour of citations
  	linktoc=page,
  	pdfborder={0 0 0},			% 	-
	breaklinks=true,			% 	- allow line break inside links
	bookmarksnumbered=true,		%
	bookmarksopenlevel=4,
	bookmarksopen=true,		%
	final=true	% = true, nur bei web-Dokument!! (wichtig!!)
}
\usepackage{bookmark}

\crefformat{lstlisting}{#2example\ #1#3}
\Crefformat{lstlisting}{#2Example #1#3}
\crefmultiformat{lstlisting}{#2examples #1#3}{; #2#1#3}{; #2#1#3}{; #2#1#3}
\Crefmultiformat{lstlisting}{#2Examples #1#3}{; #2#1#3}{; #2#1#3}{; #2#1#3}
\Crefrangeformat{lstlisting}{#3Examples #1#4--#5#2#6}
\crefrangeformat{lstlisting}{#3examples #1#4--#5#2#6}


\begin{document}
\title{\texttt{arthistory} -- \\\texttt{bib\LaTeX} for art historian\footnote{The development of the code is done at \url{https://github.com/LukasCBossert/biblatex-arthistory}: 
Comments and criticisms are welcome.
}}
\author{Lukas C. Bossert\thanks{\href{mailto:lukas@digitales-altertum.de}{lukas@digitales-altertum.de}} \and Thorsten Kemper}
\date{Version: \thisarthistoryversion{} (\thisarthistorydate)} 
 \maketitle
 \begin{abstract}
\noindent This citation-style covers the citation and bibliography rules of the Kunsthistorisches Institut der Universität Bonn.
Various options are available to change and adjust the outcome according to one's own preferences. 
The style is compatible with the English and German, since all |bibstrings| used are defined in both languages.

 \end{abstract}


\begin{multicols}{2}
\footnotesize\parskip=0mm \tableofcontents
\end{multicols}

\section{Installation}
|arthistory| is part of the distributions MiK\TeX \footnote{Website: \url{http://www.miktex.org}.} 
and \TeX Live\footnote{Website: \url{http://www.tug.org/texlive}.}~-- thus, you
can easily install it using the respective package manager. 
If you would like to
install |arthistory| manually, do the following:
Download the folder |arthistory| with all relevant files from the CTAN-server\footnote{\url{http://mirrors.ctan.org/macros/latex/contrib/biblatex-contrib/arthistory.zip}} and copy it to the \texttt{\$LOCALTEXMF} directory of
 your system.\footnote{If you don't know what that is, have a look at
\url{http://www.tex.ac.uk/cgi-bin/texfaq2html?label=tds} or 
\url{http://mirror.ctan.org/tds/tds.html}.} 
Refresh your filename database.\footnote{ 
Here is some additional information from the UK \TeX\ FAQ:
\begin{itemize}[nosep,after=\vspace{-\baselineskip} ]
  \item \href{%
    http://www.tex.ac.uk/cgi-bin/texfaq2html?label=install-where}{%
    Where to install packages}
  \item \href{%
    http://www.tex.ac.uk/cgi-bin/texfaq2html?label=inst-wlcf}{%
    Installing files \enquote{where \LaTeX /TeX\ can find them}}
  \item \href{%
    http://www.tex.ac.uk/cgi-bin/texfaq2html?label=privinst}{%
    \enquote{Private} installations of files}
\end{itemize}
}
%%introduction from biblatex-dw copied and applied. might to be rewritten.

\section{Usage}
 \DescribeMacro{arthistory}  The name of the bib\LaTeX-style is  |arthistory| which has to be activated in the preamble. 

\begin{code}
\usepackage[style=arthistory,%
          *@\meta{further options}@*]{biblatex}
\bibliography*@\marg{|bib|-file}@*
\end{code}

Without any further options the style follows the rules of the Kunsthistorisches Institut der Universität Bonn.
No additional settings are needed,
but you can change the outcome by using some options which are explained below.\footnote{For an easy and unproblematic compiling we suggest to use \hologo{XeLaTeX} or  \hologo{LuaTeX}.}

At the end of your document you can write the command |\printbibliography| to print 
the bibliography. 
Since |arthistory| supports different citations of various texts such as those of ancient authors and  modern scholars we suggest  having them listed in separate bibliographies. 
Further information can be found below %  (\cref{bibliographie}).

\section{Overview}\label{overview}
There follows a quick overview of possible options of the style |arthistory|. 
Furthermore you can -- at your own risk -- also use the conventional |bib|\LaTeX-options relating to indent, etc. 
For that please see the excellent documentation of  |bib|\LaTeX.

\subsection{Preamble options}\label{preamble_options}

\DescribeMacro{citeauthorformat}
You can chose how the name of authors or editors are displayed within your text when they are cited with \cs{citeauthor}\marg{bibtex-key}.
You can chose between the options \meta{initials}, \meta{full}, \meta{family}, \meta{firstfull}; 
cf. %\cref{citeauthorformat}.

\DescribeMacro{width}
|width={value}| defines the bibliography width between label and reference; cf. %\cref{width}.

\subsection{Entry Options}
A single bibliography entry can contain a value in its |options|-field.
Depending on the option it changes the behaviour of how that entry is cited.
Beside their distinct properties all of these options have in common that the separating comma between citation and page record is missing. 
Actually this concerns citation of ancient texts and corpora where usually the |shorthand|-field is printed in citations.

\DescribeMacro{source}
The entry is a source (e.\,g. Cicero, Plutarch, etc); %cf. \cref{source}.



\subsection{Cite commands}\label{cite-commands}
\DescribeMacro{\cite}%
As always citing is done with \cs{cite}:
\begin{code}
\cite*@\oarg{prenote}\oarg{postnote}\marg{bibtex-key}%@*
\end{code}

\meta{prenote} sets a short preliminary note (e.\,g. \enquote{Vgl.}) and \meta{postnote} is usually used for page numbers.
If only one optional argument is used then it is \oarg{postnote}.
\begin{code}
\cite*@\oarg{postnote}\marg{bibtex-key}%@*
\end{code}
The \meta{bibtex-key} corresponds to the key from the bibliography file.

\DescribeMacro{\cites}
If one wants to cite several authors or works a very convenient way is the following using the \cs{cites}-command:
\begin{code}
\cites(pre-prenote)(post-postnote)
  *@\oarg{prenote}\oarg{postnote}\marg{bibtex-key}@*%
  *@\oarg{prenote}\oarg{postnote}\marg{bibtex-key}@*%
  *@\oarg{prenote}\oarg{postnote}\marg{bibtex-key}\ldots@*
\end{code}
 
 \DescribeMacro{\footcite}
 There is also the possibility to put the citation into a footnote at once:
 \begin{code}
\footcite*@\oarg{prenote}\oarg{postnote}\marg{bibtex-key}@*
\end{code}

\DescribeMacro{\parencite}
Sometimes a citation has to be put in parentheses. 
Therefore we implemented the command \cs{parencite}:
\begin{code}
\parencite*@\oarg{postnote}\marg{bibtex-key}%@*
\end{code} 
This cite command takes care of the correct corresponding parentheses and brackets.
Especially in |@Inreference| citations the parentheses  change to (square) brackets.


\DescribeMacro{\parencites}
Of course there is also the possibility to cite several authors/works in parentheses.
This is done with \cs{parencites}:
\begin{code}
\parencites(pre-prenote)(post-postnote)%
*@\oarg{prenote}\oarg{postnote}\marg{bibtex-key}@*%
*@\oarg{prenote}\oarg{postnote}\marg{bibtex-key}@*%
*@\oarg{prenote}\oarg{postnote}\marg{bibtex-key}\ldots@*
\end{code}
 
\DescribeMacro{\textcite}
Beside the listed \cs{cite} commands above there is a third way of citing:
\cs{textcite} is useful if the author should be mentioned in the text and
the remaining components such as year and page will immediately follow in parentheses. 
\begin{code}
\textcite*@\oarg{postnote}\marg{bibtex-key}%@*
\end{code} 

\DescribeMacro{\textcites}
And again there is also a \cs{textcites} in case of several authors: 
\begin{code}
\textcites(pre-prenote)(post-postnote)%
  *@\oarg{prenote}\oarg{postnote}\marg{bibtex-key}@*%
  *@\oarg{prenote}\oarg{postnote}\marg{bibtex-key}@*%
  *@\oarg{prenote}\oarg{postnote}\marg{bibtex-key}\ldots@*
\end{code}

\DescribeMacro{\citeauthor}\DescribeMacro{\citetitle}\label{citeauthor}%
Furthermore and in addition to the ›normal‹ \cs{cite}-commands one can also cite only the author or the work title in the text and in the footnotes.
\begin{code}
\citeauthor*@\oarg{prenote}\oarg{postnote}\marg{bibtex-key}%@*
\end{code} 
  and for the works 
\begin{code}
\citetitle*@\oarg{prenote}\oarg{postnote}\marg{bibtex-key}%@*
\end{code} 

\section{Entry types}
\subsection{article}
\DescribeMacro{@article}
\begin{bibexample}[label=Schlegel1992]{{@}Article\{Schlegel1992,…\}}
@Article{Schlegel1992,
  author       = {Schlegel, Ursula},
  title        = {Ein Terracottamodell des Bartolomeo Ammannati},
  journaltitle = {Paragona/Arte},
  volume       = {43},
  pages        = {25--30},
  year         = {1992},
  number       = {503},
}
\end{bibexample}

\printbib[6em]{Schlegel1992}

\subsection{book}
\DescribeMacro{@book}
\begin{bibexample}[label=vonrosen2009]{{@}Book\{vonrosen2009,…\}}
@Book{vonrosen2009,
  author   = {von Rosen, Valeska},
  title    = {Caravaggio und die Grenzen des Darstellbaren},
  subtitle = {Ambiguität, Ironie und Performativität in der Malerei um 1600},
  location = {Berlin},
  year     = {2009},
}
\end{bibexample}
\printbib[6em]{vonrosen2009}

\subsection{exhibition catalog}
\DescribeMacro{@exhibcatalog}
\begin{bibexample}[label=AusstellungBonn2005]{{@}exhibcatalog\{AusstellungBonn2005,…\}}
@Exhibcatalog{AusstellungBonn2005,
  editor          = {Jutta Frings},
  year            = {2005},
  location        = {Leipzig},
  eventdate       = {2005/2006},
  eventsubtitle   = {Kunst und Kultur der Päpste II 1572--1676},
  eventtitle      = {Barock im Vatikan},
  eventtitleaddon = {Bonn, Kunst- und Ausstellungshalle der Bundesrepublik Deutschland, 25. November 2005 bis 19. März 2006; Berlin, Martin-Gropius-Bau, 12. April bis 10. Juli 2006},
  keywords        = {Ausstellung},
  venue           = {Bonn/Berlin},
}
\end{bibexample}
\printbib[6em]{AusstellungBonn2005}


\end{document}